% This page is hand-made. I could not make Biblatex to output the papers the way I wanted.

\begin{refsection}

This thesis is based on the following appended papers:

\begin{description}
% Biblatex \fullcite{Voronov2011} would work, but it uses maxcitenames, not maxbibnames, and there is no obvious way to change maxbibnames locally and change it back afterwards.
\item[Paper~\ref{pap:liras}.] Pfreundschuh, S., Eriksson, P., Buehler, S. A., Brath, M., Duncan, D., Larsson, R., and Ekelund, R. (2019). ``Synergistic radar and radiometer retrievals of ice hydrometeors ``, Atmos. Meas. Tech. Discuss.,  in review.
\item[Paper~\ref{pap:joint_flight}.] Pfreundschuh, S., Eriksson, P., Buehler, S. A., Brath, M., Duncan, D., Ewald, F., and Delanoë, J. (2019). ``Relating microphysical and radiometric properties of cloud hydrometeors at millimeter and sub-millimeter wavelengths'', Atmos. Meas. Tech. Discuss.,   manuscript in preparation.
\end{description}

\vspace{1cm}

\noindent Other relevant publications authored or co-authored by Simon Pfreundschuh:
\begin{description}
\normalsize
\item Ekelund, R., Eriksson, P., and Pfreundschuh, S. (2019). ``Using passive and active microwave observations to constrain ice particle models'', Atmos. Meas. Tech. Discuss, in review.
\item Hagen, J., Hocke, K., Stober, G., Pfreundschuh, S., Murk, A., and Kämpfer, N. (2019). ``First measurements of tides in the stratosphere and lower mesosphere by ground-based Doppler microwave wind radiometry'', Atmos. Chem. Phys. Discuss., in review. 
\item Duncan, D. I., Eriksson, P., and Pfreundschuh, S. (2019). ``An experimental 2DVAR retrieval using AMSR2'', Atmos. Meas. Tech. Discuss., accepted.
\item Duncan, D. I., Eriksson, P., Pfreundschuh, S., Klepp, C., and Jones, D. C. (2019). ``On the distinctiveness of observed oceanic raindrop distributions'', Atmos. Chem. Phys., 19, 6969–6984, https://doi.org/10.5194/acp-19-6969-2019.
\item Pfreundschuh, S., Eriksson, P., Duncan, D., Rydberg, B., Håkansson, N., and Thoss, A (2018). ``A neural network approach to estimating a posteriori distributions of Bayesian retrieval problems'', Atmos. Meas. Tech. , 11, 4627–4643, \\ https://doi.org/10.5194/amt-11-4627-2018.
\end{description}

\end{refsection}

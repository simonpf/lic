  %\thesistitle\\
  %Thesis for the degree of Licentiate of Engineering\\
  %Simon Pfreundschuh\\
  %Department of Space, Earth and Environment\\
  %Chalmers University of Technology
  %\vskip 1pc

\vspace{0.1cm}
\begin{abstract}
Global observations of clouds and precipitation are of great importance for
weather prediction and the monitoring of the climate. Nonetheless, the currently
available record of global observations does not constrain the properties of
clouds very well owing to the inherent limitations of the observation systems
used to produce them. The upcoming Ice Cloud Imager (ICI) microwave radiometer,
which will be launched on the next generation of European weather satellites,
will improve this situation by providing observations of clouds at
sub-millimeter wavelengths. ICI will be the first sensor of its kind to deliver
these observations, which will significantly improve the sensitivity to small
ice particles and low mass concentrations compared to currently available
microwave observations.

This thesis presents research aimed at developing knowledge and methodology
required for the modeling and interpretation of the observations that will be
provided by ICI. Two studies are presented which propose a method for measuring
distributions of ice hydrometeors from ICI-type sub-millimeter observations
combined with radar observations. 

The first study uses simulated observations to assess the potential benefits of
combining sub-millimeter radiometer observations with radar observations for the
retrieval of ice hydrometeors. It is found that the combined observations
improve the sensitivity to microphysical properties of clouds, which can help to
reduce the error in the retrieved mass concentrations for specific hydrometeor
types. Furthermore, improved sensitivity to supercooled liquid cloud is found as
an additional synergy between the active and passive observations.

The second study aims to validate the results from the first by applying the
synergistic retrieval algorithm to observations from a flight campaign. The
retrieval algorithm is found to show overall good agreement with in-situ
measured ice concentrations taking into account the sensitivity limits of the
sensors. In addition to that, indications of a signal from mixed-phase particles
are found in observations of convective updrafts.


In the two presented studies, a synergistic retrieval algorithm for ice
hydrometeors from radar and passive sub-millimeters has been developed,
characterized and validated. The method can be applied in a future satellite
mission to reduce uncertainties in global observations of clouds or used to
study cloud microphysical properties in field campaigns. In addition to that,
the presented application to field campaign data provides one of the rare
validation cases for the radiative transfer modeling involving clouds at
sub-millimeter wavelengths.

\textbf{Keywords:} Microwave remote sensing, hydrometeors, clouds, precipitation

\end{abstract}

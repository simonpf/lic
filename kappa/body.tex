%!TEX root = ../main.tex

\chapter{Introduction}

Clouds and precipitation affect live on Earth in multiple ways and on multiple
time scales. On short time scales, weather affects human activity, potentially
posing threats to transport, agriculture and lives. On longer time scales,
precipitation patterns shape ecosystems and societies while clouds influence the
global climate system through their interaction with the incoming and outgoing
electromagnetic radiation.

As historical evidence suggests, understanding and predicting weather and climate
has been a human endeavor dating back at least to the formation of the first
settled communities during the agricultural revolution. This is not surprising
considering the dependence of agricultural activity on benign weather patterns.
Today, this dependency has likely even increased as more and more branches of
human activity rely on the availability of reliable weather forecasts.

As the dramatic effects of anthropogenic climate change become more and more
apparent \cite{coronese19, grinsted19}, the importance of understanding and
predicting the Earth's changing climate is indisputable. The predicted heating
under all but the lowest emission scenario is likely to exceed $2\ \unit{^\circ
  C}$ by the end of the century \cite{collins13}. With this, global mean surface
air temperature will become higher than what has can be estimated from
reconstructions of past climate \cite{delmotte13}.This only underlines the
importance of a thorough understanding of the processes that govern climate
change on Earth.

The subject of this thesis are observational methods for clouds using microwave
sensors. These observations provide valuable information on the state of the
atmosphere for weather prediction as well as a basis for the evaluation and
improvement of climate models. The remainder of this introduction aims to give
an overview over the relevance of observations of clouds for both weather and
climate prediction and closes with a discussion of currently available
observation methods. The following chapters then provide an introduction to
physical and mathematical principles upon which these observations are founded.
Chapter~\ref{ch:clouds} provides an introduction to cloud formation and
establishes what properties of clouds can be observed.
Chapter~\ref{ch:radiative_transfer} introduces the theory of radiative transfer
which describes how radiation interact with clouds and which is necessary to
understand the observable effects that clouds have on microwave radiation.
Finally, Chapter~\ref{ch:inverse_proplems} introduces the mathematical methods
that are used to infer clouds properties from observations of clouds.

A brief remark on the terminology: Clouds and precipitation consist of particles
formed by frozen or liquid water. Hydrometeor is used in the following as a
collective term to denote either of these forms. Moreover, the adopted viewpoint
is that precipitation is a byproduct of cloud formation and therefore the terms
'observations of clouds' or 'cloud observations' are used to denote observations
of both clouds and precipitation.

\section{The role of clouds and precipitation in numerical weather prediction}

Clouds and precipitation are responsible for many of the phenomena that are
considered as weather. It should therefore be clear that their accurate
representation in numerical weather prediction (NWP) models is essential for
reliable forecasts. But clouds, more specifically observations of clouds, can
impact weather forecasts in another, more nuanced way.

Modern forecasting systems make use of satellite observations to determine an
optimal initial state from which a forecast run is started. This process is
called data assimilation. In a clear atmosphere, satellite observations from the
infrared and microwave domain can provide direct information on the temperature
and humidity field of the atmosphere. When observations are assimilated over
multiple time steps, sequences of observations can provide additional
information on wind speeds \cite{geer18}. In a cloudy atmosphere, satellite
observations still provide information on temperature and humidity, but in
addition to that the presence of clouds can provide information on the dynamical
state of the atmosphere.

Clouds form where warm and moist air is transported upwards in the atmosphere
by convection. Cloud formation thus acts as a tracer of atmospheric dynamics
that can provide valuable information on the state of the atmosphere. Indeed,
owing to recent developments in data assimilation methodology, microwave
observations sensitive to humidity and clouds have become a main contributor
to short-term forecast skill \cite{geer17}.

\section{The role of clouds and precipitation in the climate system}

The main impact of clouds on the climate system is through their interaction
with the incoming solar radiation and the outgoing long-wave radiation, which
determines the Earth's energy Budget. In addition to that, clouds form a part
of the global hydrological cycle in which they deliver water from the atmosphere
to the surface of the Earth.

\subsection{The global energy budget}

The short-wave radiation emitted by the sun that reaches the earth is the energy
source that drives the climate system. To remain at a stable temperature the
Earth needs to emit the same amount of incoming energy in the form of outgoing
long-wave radiation. From an energy balance perspective, clouds have two
opposing effects on the global energy budget. The first one is a cooling effect
caused by the reflection of incoming short-wave radiation back to space. The
second effect is a warming effect caused by the blocking of outgoing long-wave
radiation that would be emitted to space in a cloud-free atmosphere. Overall,
the cooling effect of clouds prevails, leading to net cooling effect of clouds.

\subsection{Cloud radiative effect and climate sensitivity}

Since both the cooling and warming effects of clouds are relatively strong,
changes in cloud properties or occurence may potentially cause significant
feedbacks in a changing climate. These effects, however, are difficult to
quantify. This is because the strength of their interaction with radiation
depends on where clouds form in the atmosphere as well as on their microphysical
properties such as particle phase and size distribution. Furthermore, the
formation of clouds involves size-ranges beginning at the nanometer scale of
aerosol particles on which cloud drops form up to several thousands of
kilometers, which is the size of synoptic-scale cyclones. This wide range of
size-scales is very difficult to represent in global climate models, which is
why they have to rely on approximate representations of cloud processes.

The cloud feedback is indeed the most uncertain feedback affecting the global
climate. It therefore contributes significantly to the spread in predicted
changes in global mean surface temperature observed between different GCMs.

\section{Global observations of clouds}

The two previous sections described the relevance of cloud observations for
weather and climate applications. This section provides a brief review of
common, global observation methods for clouds. The focus here is put on
satellite-based observations since these are the only ones that can provide
observations at global  scales and temporal resolutions. 

\subsection{Frequency domains}

Hydrometeors can be observed from the optical domain down to microwave frequencies
around $10\ \unit{GHz}$. The strength of their interaction with radiation is determined
by the relation of their size to the wavelength of the radiation, which is strongest
when the wavelength is close to the particle size.

In the optical and infrared, cloud droplets and ice crystals are large compared
to the wavelength of the radiation, so the interaction with a single particle is
not very strong. However, the large number concentrations of these particles
make clouds opaque at these wavelengths. This has important implications for
cloud observations using optical or infrared radiation. Since the cloud is
opaque the observed radiation is sensitive only to the upper-most regions of the
cloud. Observational techniques based on these wavelength therefore have no
direct sensitivity to precipitation and large parts of the water masses in thick
clouds.

At microwave frequencies, the wavelength of the radiation is much larger than
the size of the hydrometeors rendering microwave radiation insensitive to all
but the largest hydrometeors. The advantage of microwave observation is that
they can directly sense precipitation, which occurs in clouds that are too thick
for optical or infrared radiation to penetrate.

A considerable gap in observation wavelengths between the highest microwave
frequencies around $2\ \unit{mm}$ and the lowest infrared observations at
around $15\ \unit{\mu m}$. This gap is due in part to the strength of the
water vapor continuum in this region of the electromagnetic spectrum, which 
makes the troposphere invisible from space for large parts of it. Another reason
are technological difficulties in the development of sensors for these wavelengths.

\subsection{A look into the future}

Nonetheless, progress is underway to close this frequency gap. The Ice cloud
imager (ICI) will be flown on the upcoming second generation of European operational
meteorological satellites (Metop-SG) and will provide observations of the atmosphere
at frequencies up to $664\ \unit{GHz}$ corresponding to a wavelength of less than
$0.5\ \unit{mm}$. These observations at sub-millimeter wavelength will considerably
increase the sensitivity to small particles and lower hydrometeor masses.

For the observations provided by ICI to be useful the development of additional
know-how and methodology is required. This is mainly because the microphysical
properties of ice, i.e. their shape and size affect the observations at these
high microwave frequencies. The development of methods to infer cloud properties
such as the mass density of hydrometeors from these observations thus requires
consideration of these microphysical properties. These developments are also the main
topic for the PhD project that lead to this thesis.

%
%   .--~*teu.
%  dF     988Nx
% d888b   `8888>
% ?8888>  98888F
%  "**"  x88888~
%       d8888*`
%     z8**"`   :
%   :?.....  ..F
%  <""888888888~
%  8:  "888888*
%  ""    "**"`
%
% http://patorjk.com/software/taag/#p=display&f=Fraktur  --  Text to ASCII graphics
%
%%%%%%%%%%%%%%%%%%%%%%%%%%%%%%%%%%%%
%%%%%%%%%%%%%%%%%%%%%%%%%%%%%%%%%%%%
\chapter{The physics of clouds and precipitation}
\label{ch:clouds}

Clouds consist of large numbers of water droplets and ice crystals that are
suspended in the air. When these drops grow sufficiently in mass they
eventually fall out of the cloud to form precipitation. This chapter gives an
overview over the processes that lead to the formation of clouds and ultimately
precipitation. Moreover, the typical properties of the hydrometeors that make up
clouds and precipitation are presented. This knowledge is required to understand
the capabilities and limitations of the observational approaches in the remainder
of this thesis.

It is common to classify cloud types according to their form which is
representative of the dynamical context in which they form. Such a
classificlation is of minor relevance for the development of observation methods
of clouds and is therefore not presented here. Instead, warm and cold clouds are
distinguished. Warm clouds exist below the $0\ \unit{^\circ C}$ isotherm and
consist solely of liquid water droplets. Cold clouds extend to above the
$0\ \unit{^\circ C}$ isotherm consist at least in part of ice particles.
Although, typically, the liquid phase is present also in cold clouds, this
classification allows the slightly different formation processes of liquid and
frozen hydrometeors to be discussed separately. Moreover, also the signature of
liquid and frozen hydrometeors in remote sensing observations are farily, which
is why observation methods for different types of hydrometeors are generally
considered separtely.

This chapter begins with an introduction to phase transitions and how they are
initiated to form clouds. Then the formation of cloud is discusses separately
for warm and cold clouds. The chapter closes with a brief discussion of the general
properties of precipitation.

\section{Principles of cloud formation}

Cloud hydrometeors form when the water vapor contained in the air undergoes a
phase change from the gas phase to the liquid or frozen phase. These processes
are denoted as condensation, for the change from the gas to the liquid phase,
and deposition, for the change from the gas to the ice phase. Their inverse
processes, i.e. the change from water in the liquid respectively ice phase to
the gas phase are denoted as evaporation and resublimation.

A necessary condition for condensation to occur is that the air is
supersaturated with respect to liquid water. This means that the partial
pressure of water vapor exceeds the saturation vapor pressure with respect to
liquid water. Similarly, supersaturation with respect to the ice phase is
required for the formation of ice particles. The supersaturation required for
the formation of clouds is reached when comparably warm and moist air is lifted
in the atmosphere. The resulting adiabatic cooling of the air leads to a
decrease in the saturation vapor pressures of water and ice and the air eventually
becomes supersaturated.

When water vapor reaches supersaturation with respect to either the liquid or
ice phase it enters a metastable state. This means that although the liquid
state is energetically favorable the transition is inhibited by an energy
barrier. Due to the random nature of the movements of water vapor molecules,
some of them eventually overcome the energy barrier by forming clusters of the
new stable phase inside the metastable gas phase. In the context of phase
transitions, this process of forming clusters of the stable state inside the
metastable parent state is referred to as nucleation. More specifically, two
types of nucleation are distinguished: Homogeneous nucleation refers to the
process of forming a new, pure cluster of molecules in the stable phase whereas
heterogeneous nucleation refers to the formation of a cluster of molecules in
the stable phase on or around a cluster of a different molecular species.

After a sufficiently large nucleus has formed, it will grow due
to the condensation or deposition of water molecules as long as
its environment is supersaturated with respect to its phase.
Eventually, differences in fall velocities between particles
of different sizes will cause the particles to collide and stick
together which further accelerates particle growth.

Due to the different molecular properties of water and ice, slightly different
processes are involved in the formation of liquid cloud droplets and ice
particles, respectively. These together with the corresponding growth mechanism
are explained in more detail in the following two sections.

\section{Warm clouds}

As has been mentioned above, warm clouds are clouds that do not extend above
the $0\ \unit{^\circ C}$ isotherm and consist solely of liquid cloud droplets.

\subsection{Formation}

The formation of cloud droplets by homogeneous nucleation is highly unlikely due
to the height of the energy barrier separating the metastable gas phase from the
liquid phase. Instead, cloud droplets form through the activation of cloud
condensation nuclei (CCN). CCN are soluble aerosol particles, which take up
water molecules and grow even in environments that are not super-saturated. The
droplets which are formed by hygroscopic growth of aerosol particles are called
solution droplets. For sufficiently high supersaturations, the energy barrier
for the transition to larger cloud drops vanishes leading to immediate
condensation of all water molecules onto the droplet that are available in its
surroundings. The theory describing the activation of CCN and their growth
to cloud droplets is know as Köhler theory \cite{kohler36}.

\subsection{Growth processes}

The condensation of water molecules onto the newly formed cloud causes a
gradient in the concentration of water molecules initiating  a diffusive flow
of water vapor towards the droplet. The water vapor flowing towards the droplet
condenses onto it making it grow in size and mass. This process is called
growth by diffusion and condensation. The rate of diffusional growth decreases
with increasing droplet radius. From sizes around $10$ or $20\ \unit{\mu m}$ and
up therefore another growth process takes over.

When the cloud droplets have grown sufficiently in mass, differences in fall
speed between cloud droplets of different size as well as turbulence may cause
droplets to collide. If these droplets coalesce the resulting droplet will have
grown compared to the two colliding droplets. The resulting, larger particle
will fall even faster down through the cloud. Heavier particles typically are
also more efficient in collecting other cloud droplets. This illustrates why
collision-coalescence is a very efficient growth process. Only
collision-coalescence can explain the onset of rain only 20-30 minutes after the
formation of a cumulus cloud which can be observed in the atmosphere.

\section{Cold clouds}

Cold clouds extend to above $0\ \unit{^\circ C}$ isotherm and are characterized
by the presence of frozen hydrometeors.


\subsection{Formation}

In contrast to liquid droplets, both homogeneous and heterogeneous nucleation
are relevant for the formation of ice particles in the atmosphere. In agreement
with Oswald's rule of stages, homogeneous nucleation occurs only through the liquid
phase due to the prohibitively high energy barrier associated with the formation
of an ice nucleus directly from the vapor phase. This means that ice particles
are formed by homogeneous nucleation through the formation of an ice nucleus
inside an existing cloud or solution droplet and subsequent complete freezing of
the droplet. Nonetheless, the energy barrier for homogeneous nucleation of ice
particles remains sufficiently high so that these processes occur only at
temperatures below $-36\ \unit{^\circ C}$ for solution droplets and
$-38\ \unit{^\circ C}$ for cloud droplets.

Heterogeneous nucleation of ice particles involves aerosol particles, so called
ice nucleating particles (INP), that provide a surface onto which the water
molecules can form aggregates with ice-like structure. Heterogeneous nucleation
is thought to occur both indirectly through the liquid phase as well as directly
from the gas to the ice phase. Heterogeneous freezing, that is heterogeneous
nucleation from the liquid phase, occurs when a cloud droplet or a solution
droplet comes in contact with an INP upon which the droplet freezes.
Heterogeneous freezing may also occur when water vapor condenses directly onto
the INP followed by freezing of the liquid nucleus formed on the INP.

Alternatively, heterogeneous nucleation may occur directly from the vapor phase
by deposition of water molecules directly onto the INP. It is, however, still
debated whether this process really occurs directly from the vapor phase or whether
an intermediate liquid nucleus is formed on the INP.

\subsection{Growth processes}

In principle, the growth processes for ice particles are the same as for liquid
droplets. However, due the potential coexistence of particles in the liquid
phase these processes have slightly different characteristics as will be
explained below.

Due to the lower saturation vapor pressure of ice compared to that of water, a
newly-formed ice nucleus experiences a much higher ratio of supersaturation then
a cloud droplet would. Because of this, the diffusional growth of ice crystals
is much faster than that for cloud droplets. The rapid growth of the ice
particles will deplete the surrounding air of water vapor. This depletion may
cause the environment to become sub-saturated with respect to water but
saturated with respect to ice. If this is the case, potentially present supercooled
cloud droplets evaporate and their molecules deposit onto the ice particles. This is
known as the Wegener-Bergeron-Findeisen process.

Similar as for cloud droplets, an ice crystal that grows sufficiently in mass
eventually starts to sediment. Size differences between different particles as
well as turbulence may cause particles to collide and potentially stick together
to form larger particles thus initiating growth by accretion. Accretion is the
general term for the growth of two hydrometeors caused by a collision and
resulting in permanent union of the two particles. For ice particles growth by
accretion can happen in two ways: The collision of two ice particles, called
aggregation, or the collision of an ice particle with a liquid particle which
freezes upon contact, called riming. Aggregation produces aggregates of
snow crystals that, if they don't melt, fall to the ground in the form of snow.
Particles produced by riming a known as graupel, when their diameter remains below
$2.5\ \unit{mm}$, and hail for sizes above that.

\subsection{Ice habits}

Ice crystals exhibit a fascinating range of different forms. Their crystal
structure depends on the thermodynamic conditions of their formation. This is
due to the dependence of the surface tension on the temperature and
supersaturation as well as the tendency of the shape to minimize this surface
tension.

A common grouping of ice particles is into pristine ice crystals, aggregates and
rimed particles. Common shapes for ice crystals are plates and columns. But also
other shapes such as dendrites, stellar plates or needles can be observed. Snow
aggregates are usually made up of 10 to 100 or more single crystals. They often
consist of dendrites and thins plates. Finally, rimed particles are typically
spherical with densities slightly lower than that of solid ice due air
inclusions.

\section{Precipitation}




\chapter{Microwave radiative transfer in the atmosphere}
\label{ch:radiative_transfer}

The underlying physical mechanism that allows cloud and precipitation to be
remotely sensed is their interaction with electromagnetic radiation that can be
measured from afar using suitable detectors. These interactions are described by
the theory of radiative transfer. Since this theory is essential for the
understanding and development of retrieval methods, this section provides an
introduction to the radiative transfer of microwaves in the atmosphere.
Particular focus is put on the interaction of radiation with clouds. This
presentation is mostly based on the more comprehensive texts by
\textcite{thomas02, mischenko02, wallace06}.

\section{The theory of radiative transfer}

Radiative transfer theory describes radiation as monochromatic beams that
transport radiant energy through the atmosphere. One of the fundamental
quantities of the theory is the spectral intensity $I_\nu$ defined as the rate
at which a beam of angular extent $d\omega$ propagating into direction
$\hat{\vec{n}}$ of frequency $\nu$ transports energy through an infinitesimal
area $dA$:
\begin{align}\label{eq:spectral_intensity}
  I_\nu = \frac{d^5E}{\cos(\theta) dA\ dt\ d\omega\ d\nu}
\end{align}
In addition to that, a monochromatic beam of radiation has a polarization state,
which describes how the energy flux is split up between the two components of
the electric field perpendicular to the propagation direction as well as their
respective phase. The intensity of a beam and its polarization state are
described by the stokes vector
\begin{align}
  \vec{I} &= \left [ \begin{array}{c}I_\nu \\ D \\ Q \\ U\end{array} \right ]
\end{align}
The four components $I, D, Q$ and $U$ of the Stokes vector fully characterize an
electromagnetic plane wave to the extent it that it can be measured using
traditional detectors. This means that all its measurable quantities can be derived from the
corresponding stokes vector.

The stokes vector can be directly related to the
electromagnetic field strength of an electromagnetic plane wave allowing it to
be derived from the more fundamental theory of electromagnetism. The key
advantage of radiative transfer theory, however, is that it allows a simplified
treatment of the problems relevant to atmospheric remote sensing which are too
complex to be solved by direct application of the laws of electrodynamics.

\subsection{Interactions with matter}

The Stokes vector provides a full description of the radiation measured by any
remote sensing instrument. To model the radiation reaching the detector, a
suitable description how this radiation is created as well as how it changes as
it propagates through the atmosphere is required. A common approach in radiative
transfer theory is to distinguish three fundamental types of such interactions
of radiation with matter: The emission of radiation, its absorption, and the
scattering of absorption away from its propagation path.

\subsubsection{Emission}

At temperatures above absolute zero, all matter emits radiation through the
process of thermal emission. Thermal emission occurs when matter transitions
from a quantum mechanical state of higher energy to one of lower energy which
causes the surplus of energy to be emitted in the form of radiation. When
considering radiation in the lower atmosphere, the relevant emitters of
radiation are the ocean or land surface as well as gas molecules or suspended
particles.

A fundamental concept for the description of emission is that of a black body.
A black body is a piece of matter that absorbs all incoming radiation. At a
given temperature $T$, the emission of a black body is isotropic and polarized.
Its spectral intensity is given by Planck's law:
\begin{align}
  B_\nu(\nu, T) &= \frac{2h\nu^3}{c^2}\frac{1}{\exp ( \frac{h\nu}{k_B T} )- 1}
\end{align}
where $c$ is the speed of light in the medium, $h$ is the Planck constant and
$k_B$ is the Boltzmann constant. The stokes vector describing the emission from
a black body is given by
\begin{align}
  \vec{I} &= \left [ \begin{array}{c}B_\nu \\ 0 \\ 0 \\ 0\end{array} \right ]
\end{align}

The concept of the black body is used to define the emission from other forms
of matter using the emissivity vector $\vec{\epsilon}$:
\begin{align}
 \vec{I} &= \vec{\epsilon} \cdot B_\nu
\end{align}
The main difference between the treatment of emission from a volume compared
to that of a surface is the unit of the emission vector $\vec{\epsilon}$. For
a volume, it is defined per unit length of the path through the volume, while
for a surface this is not necessary.

Due to the distinct orientation that surfaces have with respect to the viewing
geometry, the emissivity vector generally depends on the incidence angle. For
particles this is generally also the case, but since most particles in the
atmosphere are randomly oriented it is often neglected. Although black-body
radiation is unpolarized, emission from general emitters can be polarized. An
important example is the ocean surface, which is highly polarized around the
Brewster angle at $53^\circ$.

\subsubsection{Absorption}

Absorption refers to the process of radiation being converted into internal
energy of the matter it interacts with. Mathematically, absorption is described
by an absorption vector $\vec{\alpha}$, defined as the fraction of the incoming
radiation that is absorbed along an infinitesimal distance $ds$ along the
propagation path:
\begin{align}
\vec{I}_\text{absorbed} &= (\vec{\alpha} \cdot\ ds) \odot \vec{I}
\end{align}
Here $\odot$ denotes the element-wise product of the absorption vector and
the stokes vector $\vec{I}$ of the incoming radiation. Absorption may be
understood as the inverse process of thermal emission. Formally, this is
expressed by Kirhoff's  law of radiation
\begin{align}
  \vec{\alpha} &= \vec{\epsilon}
\end{align}
This law is applicable to all matter in the atmosphere given that it is in a
state of local thermal equilibrium (LTE). LTE occurs when the density of matter
is sufficiently high so that the population frequencies of energy states above
the ground state are determined by thermal collisions rather than the absorption
of radiation. This decouples the emission of radiation from the radiation field,
allowing the simplified treatment of matter as thermal emitters with the
emission rates independent of the radiation field. LTE is a valid assumption for
radiative transfer in the lower atmosphere.

\subsubsection{Scattering}

When a beam of radiation impinges upon a particle, their interaction may cause a
deviation of parts of the beam from the original propagation path. To first
order, scattering decreases the intensity of the beam. This particular process
is referred to as single scattering. As it propagates through the atmosphere,
the intensity of a beam is decreased by the effects of absorption and single
scattering. The combination of these two processes is referred to as
attenuation.

However, as the rate of scattering increases, also the effect of multiple
scattering has to be taken into account. Multiple scattering occurs when energy
from other beams scattered into the line of sight causes an increase in the
intensity of the considered beam.

Mathematically, the scattering of a beam of light propagating in direction
$\vec{n}$ into the direction $\vec{\hat{n}}$ is described by the phase
matrix $\mat{Z}(\vec{n}, \vec{\hat{n}})$:
\begin{align}
  \vec{I}_\text{scattered}(\vec{\hat{n}}) &= \mat{Z}(\vec{\hat{n}}, \vec{n}) \vec{I}(\vec{n})
\end{align}
The combined, attenuating effects of scattering and absorption are give by
the attenuation matrix $\mat{K}$, given by the sum of the absorption vector
$\vec{a}$ and the fraction of radiation scattered away from the propagation
path:
\begin{align}
  \vec{K} &= \vec{\alpha} + \int_{\vec{\hat{n}}} d\vec{\hat{n}}\ \mat{Z}(\vec{\hat{n}}, \vec{n})
\end{align}

\subsection{The radiative transfer equation}

The previous section introduced the fundamental interactions of radiation
with matter and how they are described mathematically in radiative transfer
theory. Combining the three processes of emission, absorption and scattering,
the change that a beam undergoes as it travels a distance $ds$ along its
propagation path through the atmosphere is described the vectorized radiative
transfer equation (VRTE):
\begin{align}\label{eq:vrte}
  \frac{d\vec{I}}{ds} &=
  -\mat{K}\vec{I} + \vec{\alpha} \cdot B_\nu(T) + \int_{\Omega} d\omega \ \mathbf{Z}(\vec{n}, \vec{\hat{n}}) I(\vec{\hat{n}}).
  \end{align}
By solving Equation~(\ref{eq:vrte}) the radiation field for an arbitrary atmosphere can
be computed. What is required for this are the values of temperature, absorption vector
$\vec{\alpha}$ and phase matrices $\mat{Z}$ throughout the atmosphere as well as a
suitable method for solving the radiative transfer equation. The values of $\vec{\alpha}$
and $\mat{Z}$ describe how a specific volume element of the atmosphere absorbs and scatters
radiation. Their values therefore depend on the concentrations of gases and particulate
matter in the atmosphere.

Approximate values of $\vec{\alpha}$ and $\mat{Z}$ for different materials in
the atmosphere can be measured experimentally or in special cases even derived
from first principles. Typically they depend on local properties of the
atmosphere such as temperature, pressure or concentration of gases or particles.
Numerical models for $\vec{\alpha}$ and $\mat{Z}$ together with the VRTE thus
allow the radiation field observed by remote sensing intruments to be related to
the state of the atmosphere. Together with the methods described in
Section~\ref{ch:inverse_problems}, this forms the basis of atmospheric
remote sensing.

\subsection{Long-wave radiative transfer in the atmosphere}

This thesis focuses on observations of the atmosphere at microwave and sub-millimeter
wavelengths. The specific properties of this frequency domain allow for a number
of simplifications, that can greatly simplify the solution of the VRTE in
the atmosphere.

At microwave frequencies, all relevant emission of radiation is due to thermal
emission from matter in the atmosphere. Emission from the sun is irrelevant at
these wavelengths. Furthermore, due to the long wavelengths of the radiation,
scattering by molecules of gases can be neglected. In this case, the integral in
Eq.~(\ref{eq:vrte}) disappears and it can be solved efficiently by integration
along the line of sight.

In the presence of clouds, solving the radiative transfer equation becomes
more complicated and computationally more demanding as it requires solving
for the whole radiation field instead of solving the VRTE along a single
beam.

\section{Microwave observations of clouds and precipitation} 

Clouds and precipitation can be observed at microwave frequencies using
active as well as passive observation techniques.

\subsection{Radar observations}


\subsection{Passive observations}


%  d8Nu.  9888c
%  88888  98888
%  "***"  9888%
%       ..@8*"
%    ````"8Weu
%   ..    ?8888L
% :@88N   '8888N
% *8888~  '8888F
% '*8"`   9888%
%   `~===*%"`
%
%%%%%%%%%%%%%%%%%%%%%%%%%%%%%%%
%%%%%%%%%%%%%%%%%%%%%%%%%%%%%%%
\chapter{Inverse problems}
\label{ch:inverse_problems}

The previous chapter introduced the general properties of clouds and  their
effect on electromagnetic radiation which can be used to observe them. This
chapter describes the mathematical methods that can be used infer properties
of clouds from their signatures in the observed electromagnetic radiation.
Mathematically, this problem can be formulated as an inverse problem. This is
because inferring the cloud properties from the observations may be viewed as the
inverse of the problem of predicting the observations from the cloud properties,
which can be solved using radiative transfer. This, the problem of solving
the radiative transfer equation in the presence of cloud hydrometeors, is referred
to as the forward problem.

\section{Formulation}

Mathematically, the inverse problem is formulated as follows: Let $\vec{x} \in
\mathbb{R}^n$ be an arbitrary vector from the state space $\mathbb{R}^n$. Here
it is assumed that the vector $\vec{x}$ describes the properties of the observed
cloud. The space $\mathbb{R}^n$ is the space of all possible cloud
configurations that could be observed. The interaction of the cloud described by
the vector $\vec{x}$ interacts with the radiation that is measured by an
observation system, producing the observation vector $\vec{y}$. It is assumed
here that a so called forward $\mathbf{F}: \mathbb{R}^n \rightarrow
\mathbb{R}^m$ exists that allows computing the observation $\mathbf{F}(\vec{x})$
corresponding to any given state vector $\vec{x}$. The inverse problem
now consists of determining the state vector $\vec{x}$ corresponding to
a given observation $\vec{y}$ thus inverting the forward model $\mathbf{F}$.

The general difficulty with inverse problems is that they do not admit
a unique solution. This is because, at least in atmospheric remote sensing, the
problem is generally underconstrained. This means that the amount of information
in the observations $\mathbf{y}$ is not sufficient to uniquely determine a state
$\mathbf{x}$. Examples are different cloud configurations that result in the
same measurement vector such as for example a low-level cloud covered by an
opaque high-level cloud. From the measurement vector $\vec{y}$ alone it is
impossible to determine a unique state $\vec{x}$ as it will be the same
independent of the presence of properties of the low-level cloud.

Simultaneously to being underconstrained, the problem may be over constrained.
This happens if different components of the measurement vector provide seemingly
contradictory information on the measurement state $\vec{x}$ due errors random
errors in $\vec{y}$.

\section{Solution}

A common approach in atmospheric remote sensing to solving inverse problems
is the application of Bayesian statistics. Instead of searching a unique solution
vector $\vec{x}$ to the inverse problem, the solution is found in the form of a
probability distribution that describes how likely it is that any of the elements
of the state space have produced the given observation $\vec{y}$.

The approach that will be presented in the following is known as the optimal estimation
method (OEM, \textcite{rodgers00}). The method makes three basic assumptions in order
to solve the inverse problem:
\begin{enumerate}
\item That the forward model $\mathbf{F}$ is linear or only weakly non-linear
\item That the knowledge available about $\vec{x}$ can be described by a Gaussian distribution
  with mean $\vec{x}_a$ and covariance matrix $\mat{S}_a^{-1}$
\item That the errors affecting $\mathbf{y}$ are Gaussian with covariance matrix $\mat{S}_\epsilon$
\end{enumerate}

Under these assumption both the a priori distribution for  $\vec{x}$ as well as the conditional
probability of observing the measurement $\vec{y}$ given the state $\vec{x}$ are Gaussian:
\begin{align}
p(\vec{x}) &= \frac{1}{(2 \pi)^{\frac{-n}{2}} \text{det}(\mat{S}_a)^{-\frac{1}{2}}} \exp \left \{ -\frac{1}{2}(\vec{x} - \vec{x}_a)^T\mat{S}_a^{-1}(\vec{x} - \vec{x}_a ) \right \}\label{eq:px}\\
p(\vec{y} | \vec{x}) &= \frac{1}{(2 \pi)^{\frac{-m}{2}} \text{det}(\mat{S}_e)^{-\frac{1}{2}}} \exp \left \{-\frac{1}{2}(\vec{x} - \mathbf{F}(\vec{x}))^T\mat{S}_e^{-1}(\vec{x} - \mathbf{F}(x)) \right \}\label{eq:pyx}
\end{align}

In the Bayesian framework the solution of the inverse problem is simply the a posteriori distribution
$p(\vec{y} | \vec{x})$ of $\vec{x}$ given the observation vector $\vec{y}$. It is found by applying
Bayes theorem
\begin{align}
  p(\vec{y} | \vec{x}) &= \frac{p(\vec{y} | \vec{x})p(\vec{x})}{p(\mathbf{y})} \\
  &\propto  p(\vec{y} | \vec{x})p(\vec{x})
\end{align}
to the probabilities  (\ref{eq:px}) and (\ref{eq:pyx}).

As a specific solution of the retrieval problem, generally the most likely state is chosen,
denoted as the maximum a posteriori (MAP) estimator for $\vec{x}$. It can be found by minimizing
the log-likelihood of the posterior distribution, which has the form:
\begin{align}\label{eq:oem_cost}
  -\mathcal{L} \propto (\vec{F}(\vec{x}) - \vec{y})^T\mat{S}_\epsilon^{-1}(\vec{F}(\vec{x}) - \vec{y})
  (\vec{x} - \vec{x}_a)^T \mat{S}_a^{-1} (\vec{x} - \vec{x}_a)
\end{align}

Solving the retrieval problem has thus been reduced to minimizing the negative log
likelihood of the posterior distribution. When the forward model is non-linear, minimizing
(\ref{eq:oem_cost}) must be performed iteratively using suitable optimization methods such
as the Gauss-Newton or Levenberg-Marquardt methods \cite{boyd04}.

\section{Error estimation}

When the forward model is approximately linear around the retrieved state $\vec{x}$, the covariance
of the a posteriori distribution is given by
\begin{align}
  \mathbf{S} = \left (\mat{K}^T \mat{S_\epsilon}^{-1} \mat{K} + \mat{S}_a^{-1} \right)^{-1}.
\end{align}
Here $\mat{K}$ is the Jacobian of the forward model evaluated at the retrieved state $\vec{x}$.


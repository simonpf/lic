%!TEX root = ../main.tex

\chapter{Introduction}

Clouds and precipitation affect life on Earth in various ways and on multiple
time scales. On short time scales, weather affects human activity, potentially
posing threats to transport, agriculture and lifes. On longer time scales,
precipitation patterns shape ecosystems and societies while clouds influence the
global climate by means of their interaction with the incoming and outgoing
electromagnetic radiation.

Understanding and predicting weather and climate has been a human endeavor
dating back at least to the formation of the first settled communities during
the agricultural revolution \parencite{hellmann08}. This is easily understood
considering the dependence of agricultural activity on benign weather patterns.
Today, this dependency may have even increased as more and more branches of
human activity rely on the availability of accurate weather forecasts.

With the dramatic effects of anthropogenic climate change becoming more and more
apparent \cite{coronese19, grinsted19}, it is indisputable that a firm
understanding of the Earth's climate system is critical to ensure a safe passage
into this uncertain future. The predicted heating under all but the lowest
emission scenarios is likely to exceed $2\ \unit{^\circ C}$ by the end of the
century \cite{collins13}. With this, global mean surface air temperature will
likely surpass even the highest temperatures found in reconstructions of the
climate of the past two million years \cite{delmotte13}. To ensure effective
adaptation to these drastic changes, comprehensive observation and modeling of
the Earth's climate is indispensable.

The subject of this thesis are observational methods for clouds and
precipitation using microwave radiation. These observations provide a monitoring
system for the climate on Earth and are crucial ingredients in today's weather
forecasting systems. In addition to this, they provide a reference for the
evaluation and improvement of climate models. More specifically, the research
presented here focuses on observations of clouds at sub-millimeter wavelengths.
Such observations will become available with the upcoming second generation of
European operational weather satellites (Metop-SG). The aim of the presented
research is to develop the methodology and know-how, which is necessary to make effective
use of these novel observations. The two presented projects focus on the
combination of observations from passive sub-millimeter radiometers with those
from a radar and the question how these can be used to more accurately determine
the properties of clouds.

The remainder of this section provides an overview over the relevance of
observations of clouds for both weather and climate applications and closes with
a discussion of currently available observation methods. The following chapters
then provide an introduction to the physical and mathematical principles upon
which these observations are founded. Chapter~\ref{ch:clouds} provides an
introduction to cloud formation and establishes what properties of clouds can be
observed. Chapter~\ref{ch:radiative_transfer} introduces the theory of radiative
transfer which is necessary to understand the observable effects that clouds
have on microwave radiation. Chapter~\ref{ch:inverse_problems} introduces the
mathematical methods that are used to infer relevant properties of clouds from
remote sensing observations.

A brief note on the terminology: The term ``hydrometeor'' will be used frequently in
the following chapter to denote the aqueous particles that make up clouds and
precipitation. Moreover, the viewpoint adopted here is that precipitation is a
byproduct of cloud formation and therefore the terms \textit{observations of
  clouds} or \textit{cloud observations} are used to denote observations of both
clouds and, if present, precipitation.

\section{The role of clouds and precipitation in numerical weather prediction}

Clouds and precipitation are responsible for many of the phenomena that are
considered as weather. The most prominent example for this are certainly storms
which can bring strong winds and heavy precipitation. It is clear that accurate
representation clouds in numerical weather prediction (NWP) models is an
essential requirement for reliable forecasts. But clouds, more specifically
observations of clouds, can impact weather forecasts in another, more nuanced
way.

The forecasting systems which are used to generate weather forecasts make use of
satellite observations to determine an optimal initial state from which a
forecast run is started. This process is called data assimilation. In a clear
atmosphere, satellite observations from infrared and microwave sensors provide
direct information on the temperature and humidity field of the atmosphere. By
assimilating observations over multiple time steps, these observation can
provide additional information on the dynamical state of the atmosphere
\cite{geer18}.

Clouds form where warm and moist air is transported upwards in the atmosphere.
Because of this relation to the dynamics of the atmosphere, clouds act as a
tracer from which a data assimilation system can extract valuable
information on the state of the atmosphere. Indeed, owing to recent developments
in data assimilation methodology, microwave observations sensitive to humidity
and clouds have become a main contributor to short-term forecast skill
\cite{geer17}.

\section{The role of clouds and precipitation in the climate system}

Clouds are an integral part of the global hydrological cycle in which they
deliver water from the atmosphere to the surface of the Earth. They are also
tightly coupled to the dynamics of the atmosphere through the effects of latent
heat and the modulation of atmospheric heating profiles \parencite{bony15}. Their
principal impact on the climate system, however, is through their interaction
with the incoming solar radiation and the outgoing long-wave radiation.

\subsection{The global energy budget}

The short-wave radiation emitted by the sun that reaches the earth is the energy
source that drives the climate system. To remain at a stable temperature, the
Earth needs to emit the same amount of incoming energy in the form of outgoing
long-wave radiation. From an energy balance perspective, clouds have two
opposing effects on the global energy budget: Firstly, a cooling effect caused
by the reflection of incoming short-wave radiation back to space, and secondly,
a warming effect caused by the blocking of outgoing long-wave radiation that
would be emitted to space in a cloud-free atmosphere. Overall, the reflection
of incoming sunlight dominates, so that clouds exercise a net cooling effect on
the climate system.

\subsection{Cloud radiative effect and climate sensitivity}

Since both the cooling and warming effects of clouds are relatively strong,
changes in cloud properties or occurrence have the potential to cause significant
feedbacks in a changing climate. These effects, however, are difficult to
quantify. This is because the strength and the type of the interaction between
clouds and the incoming or outgoing radiation depends on where these clouds form
in the atmosphere. In addition to this, they are affected also by the
microphysical properties of the cloud, e.g. particle shape, phase and size
distribution.

An additional difficulty in representing the effect of clouds in the general
circulation models (GCMs) that are used for climate prediction is that their
formation and evolution involves processes at size-ranges beginning at the
nanometer scales of aerosol particles on which cloud drops form up to several
thousands of kilometers, which is the size of synoptic-scale cyclones. Since
such a range of size-scales cannot be represented explicitly in a GCM, models
have to rely on approximate representations of cloud processes.

Regarding predictions of future climate, it is indeed the strength of feedback
effects involving clouds that exhibit the largest spread between different
models. Clouds are therefore recognized as one of the major sources of
uncertainty in current climate models and contribute significantly to the spread
in predicted changes of global mean surface temperature that is observed between
different GCMs \parencite{boucher13}.

\section{Global observations of clouds}
\label{sec:observations}

The two previous sections described the relevance of cloud observations for
weather and climate applications. From this we now turn to a brief review of
currently available observation methods for clouds. The focus here is on
satellite-based systems, since these are the only ones that can simultaneously
provide observations at global scale and sufficiently high temporal sampling.

\subsection{Frequency domains}

Hydrometeors can be observed from the optical domain down to microwave
frequencies around $10\ \unit{GHz}$. The strength of their interaction with
radiation depends on the relation of their size to the wavelength of the
radiation. The interaction is strongest when the wavelength is close to the
size of the particles.

In the optical domain, cloud droplets and ice crystals are large compared to the
wavelength of the radiation, which is why they efficiently scatter of radiation
at these wavelengths. The high number concentrations of these particles present
in clouds make most of them opaque. For observations in the infrared, the
wavelength approaches the size of the smallest hydrometeors around $5\ \unit{\mu
  m}$. Since this further increases the strength of the interaction between the
cloud particles of the radiation, clouds become even more opaque at these
frequencies.

This has important implications for cloud observations using optical or infrared
radiation. Due to the opacity of most clouds, the observed radiation is
sensitive only to the upper-most regions of the cloud. Observational techniques
based on these wavelengths have therefore no direct sensitivity to precipitation
and large parts of the mass of thick clouds.

At microwave frequencies, the wavelength of the radiation is much larger than
the size of the hydrometeors, which renders it insensitive to all but the
largest hydrometeors. The advantage of microwave observations, however, is that
they can directly sense precipitation, which occurs at the base of and below
clouds that are too thick for optical or infrared radiation to penetrate.

An example of this is given in Fig.~\ref{fig:modis_gmi}, which shows simultaneous
observations of a tropical cloud system in the optical and the microwave
domain. Each panel shows a true-color composite overlaid with the
isolines of microwave observations at different wavelengths. Considering the
lowest microwave frequency (Panel (a)), $10\ \unit{GHz}$, it can be seen that
the microwave observation exhibits structure in those regions where the optical
image is saturated, i.e. only the white cloud top is visible. As the frequency
increases (Panel (b) and (c)), the structures observed in microwave image become
more complex, since the sensitivity of the radiation to hydrometeors as well as
water vapor increases. At $183\ \unit{GHz}$ (Panel (d)), the sensitivity to
atmospheric humidity is so large that the radiation cannot sense the lower
troposphere anymore. A signal from the cloud is therefore only observed where it
reaches high enough altitudes for its ice particles to scatter the up-welling thermal
radiation. We will turn back to this scene for a more detailed discussion of the
observable effects of clouds and precipitation on microwave radiation in
Chapter~\ref{ch:radiative_transfer}, which explains the principles of radiative
transfer in the presence of clouds and precipitation.

\begin{figure}
\includegraphics[width = \textwidth]{notebooks/modis_gmi.png}
\caption{The background image in each panel shows true-color composites of
  observations from the MODIS sensor. Plotted on top are the contours of the
  radiances of different channels of the GMI \parencite{draper15} radiometer.
  Dashed, gray lines mark the boundaries of the GMI swath.}
\label{fig:modis_gmi}
\end{figure}


\subsection{Active and passive sensors}

So far, only observations from passive sensors have been discussed. Passive
sensors are sensors that measure the radiation emitted by the sun or the earth
system. The signal observed by passive sensors at wavelengths shorter than
$5\ \unit{\mu m}$ is dominated by sunlight that was scattered or reflected by
the Earth or its atmosphere. At longer wavelengths, i.e. in the thermal infrared
and microwave regions of the electromagnetic spectrum sensors measure the
thermal emission from the Earth's surface or atmosphere.

In contrast to this, active sensors emit radiation themselves and measure the
amount of if that is reflected back to the sensor. The two types of active
sensors that are relevant for studying clouds are radars and lidars. The main
difference between radar and lidars are the frequencies at which they operate:
While a radar uses microwaves, lidars operate at infrared, optical or UV
wavelengths. The advantage of active sensors is that they can provide vertically
resolved observations and generally have greater sensitivity than their passive
counterparts. The vertical resolution, however, generally comes at the cost of a
decreased horizontal coverage. Active sensors therefore typically provide significantly
lower temporal sampling of specific locations on Earth than their passive counterparts.

As an example for observations from an active sensor, radar observations of the
tropical scene considered above are displayed in Fig.~\ref{fig:modis_cloudsat}.
Shown in Panel (a) of the figure are once again the passive observations in the
optical together with the ground track of the radar observations. The radar
observations along the ground track are displayed in Panel (b) of the figure.
The radar observations nicely reveal the intricate vertical structure of the
cloud. In the northernmost part of the cloud system, the cloud reaches up to
altitudes of almost $15\ \unit{km}$, a sign of strong convective activity.
Moreover, a radar signal that reaches down to the surface is observed almost
everywhere in the cloud, indicating the presence of precipitation.

\begin{figure}
\includegraphics[width = \textwidth]{notebooks/modis_cloudsat.png}
\caption{Radar observations of the tropical cloud system shown in
  Fig.~\ref{fig:modis_gmi}. Panel (a) displays once again the observations from
  the optical domain together with the ground track of the radar observations
  (dashed, red line). Panel (b) displays the radar reflectivities observed by
  the Cloud Profiling Radar on the CloudSat satellite \cite{tanelli08} along the
  track marked in Panel (a).}
\label{fig:modis_cloudsat}
\end{figure}
\subsection{A look into the future}

There is a considerable gap in the available wavelengths for the observations of
clouds between the shortest microwave wavelengths around $2\ \unit{mm}$ and the
longest infrared wavelength at around $15\ \unit{\mu m}$. This gap is partly
due to the strength of the water vapor continuum in this region of the
electromagnetic spectrum, which effectively makes the troposphere invisible for
large parts of the range, as well as technological difficulties in the
development of sensors for these wavelengths.

Nonetheless, progress is underway to narrow down this wavelength gap. The Ice Cloud
Imager (ICI, \textcite{buehler12}) is a new type of sensor that will be flown on the upcoming second
generation of European operational meteorological satellites (Metop-SG). It will
provide observations of the atmosphere at frequencies up to $664\ \unit{GHz}$
corresponding to a wavelength of less than $0.5\ \unit{mm}$. These observations
at sub-millimeter wavelengths will considerably increase the sensitivity to small
particles and lower hydrometeor masses. Moreover, ICI is expected to also provide
increased sensitivity to the shape of ice particles.

Since ICI will be the first sensor of its kind to provide observations of clouds
at sub-millimeter wavelengths, the development of new methodology and know-how
is required to fully exploit their potential. ICI's sensitivity to the
microphysical properties of hydrometeors requires improvements of their
representation in radiative transfer models and better knowledge of their
characteristic distribution in the atmosphere to accurately simulate and
understand the impact of ice particles on sub-millimeter observations.
Contributing to these preparations for the ICI sensor are the principal aims of
the research project that led to this thesis.

%
%   .--~*teu.
%  dF     988Nx
% d888b   `8888>
% ?8888>  98888F
%  "**"  x88888~
%       d8888*`
%     z8**"`   :
%   :?.....  ..F
%  <""888888888~
%  8:  "888888*
%  ""    "**"`
%
% http://patorjk.com/software/taag/#p=display&f=Fraktur  --  Text to ASCII graphics
%
%%%%%%%%%%%%%%%%%%%%%%%%%%%%%%%%%%%%
%%%%%%%%%%%%%%%%%%%%%%%%%%%%%%%%%%%%
\chapter{The physics of clouds and precipitation}
\label{ch:clouds}

Clouds consist of large numbers of water droplets and ice crystals that are
suspended in the air. When these drops grow sufficiently in mass they eventually
fall out of the cloud to form precipitation. This chapter gives an overview over
the processes that lead to the formation of clouds and ultimately precipitation.
Moreover, the typical properties of the hydrometeors that make up clouds and
precipitation are presented. This knowledge is required to understand the
capabilities and limitations of the observational approaches considered in the
remainder of this thesis. The presentation given below is based on the book
by \textcite{lohmann16}.

The discussion of clouds presented here focuses on their microphysical
properties and therefore distinguishes between warm and cold clouds. Warm clouds
exist below the $0\ \unit{^\circ C}$ isotherm and consist solely of liquid water
droplets. Cold clouds extend  above the $0\ \unit{^\circ C}$ isotherm and consist
at least in part of ice particles. Although, typically, the liquid phase is
present also in cold clouds, this classification allows the slightly different
formation processes of liquid and frozen hydrometeors to be discussed
separately. Although it is common to further classify clouds according
to their structure and the dynamical context of their formation, this
only indirectly affects their interaction with radiation and is therefore
neglected here.

This chapter begins with a brief overview over the different hydrometeor types
and sizes present in clouds. Thereafter, phase transitions are introduced which
form the basis of cloud formation processes. This is followed by a description
of the formation and growth processes in warm and cold clouds. The chapter
closes with a brief discussion of the general properties of precipitation.

\section{Types of hydrometeors}

\input{kappa/prism}

\begin{figure}[!hbpt]
  \centering
\begin{tikzpicture}

  \clip  (-7.5, 4) rectangle ++(15, 6);
\draw[black, fill = gray!25] (5,-7) circle (12.5cm);
  \draw (4, 8) node(i_label) [anchor = west] {\tiny Ice crystals ($D_\text{max} \approx 100\ \unit{\mu m}$)};
  \draw[->] (i_label.west) -- (3, 8);
  \draw (4, 7.5) node(s_label) [anchor = north west, align=left] {\tiny Snow flakes, graupel, hail \\[-0.1cm] \tiny ($D_\text{max} \approx 2500\ \unit{\mu m}$)};
  \draw[->] (s_label.west) to [out=-180, in=90] (3, 6);

  \draw[black, fill = black] (0,8) node(ccn) {} circle (0.00025cm);
  \draw (-2, 8) node(ccn_label) [anchor = east, align=left] {\tiny Cloud condensation nucleus ($D \approx 0.05\ \unit{\mu m}$)};
  \draw[->] (ccn_label.east) -- (ccn.west);

  \draw[black, fill = black] (0,7.5) node (sd) {} circle (0.0025cm);
  \draw (-2, 7.5) node(sd_label) [anchor = east] {\tiny Solution droplet ($D \approx 0.5\ \unit{\mu m}$)};
 \draw[->] (sd_label.east) -- (sd.west);

  \draw[black, fill = white] (0,7) node (cd) {} circle (0.025cm);
  \draw (-2, 7) node(cd_label) [anchor = east] {\tiny Cloud droplet ($D \approx 5\ \unit{\mu m}$)};
  \draw[->] (cd_label.east) -- (cd.west);

  \draw[black, fill = white] (0,6) node (d) {} circle (0.5);
  \draw (-2, 6.5) node(d_label) [anchor = east] {\tiny Drizzle ($D \approx 100\ \unit{\mu m}$)};
  \draw[<-] (d.center) + (-0.6 , +0.2)  -- (d_label.east);
%
  \draw (-2, 6) node(rd_label) [anchor = east] {\tiny Rain drop ($D \approx 1000\ \unit{\mu m}$)};
  \draw[black, fill = white] (0,0) node (r) {} circle (5cm);


  \begin{scope}[shift = {(2, 7)}, tdplot_main_coords]
  \pic[shift={(0, 0, 1.5)},
    /hexagonal prism/height = 0.2,
    /hexagonal prism/diameter=0.57,
    /hexagonal prism/rotation angle = 0] {hexi};
 \pic[/hexagonal prism/height = 0.99, /hexagonal prism/diameter=0.2] {hexi};
 \end{scope}
\end{tikzpicture}
\caption{Types of hydrometeors and their sizes. The figure schematically
  displays the different particles formed inside clouds. The particles are drawn
  to scale with an enlargement factor of $1000$.}
\label{fig:hydrometeors}
\end{figure}

Liquid cloud droplets form through the activation of  solution
droplets, which are aerosols that have taken up humidity from their environment.
The typical sizes of cloud droplets range from $5\ \unit{\mu m}$ to
$20\ \unit{\mu m}$ and they are the smallest particles that can be found in
tropospheric clouds. If cloud droplets grow by colliding with other cloud
droplets they eventual become drizzle drops, which have typical sizes
starting from $100\ \unit{\mu m}$ and are heavy enough to fall out of the cloud
and reach the ground. When sufficiently many other droplets are available,
drizzle droplets can continue to grow to form rain drops which have sizes from
$1$ to about $4\ \unit{mm}$.

Ice crystals are the frozen counterpart to cloud droplets and typically have
sizes around $100\ \unit{\mu m}$ but can become significantly larger and even
fall out of the cloud to reach the ground. Graupel is created when ice crystals
collide with cloud droplets that freeze upon the ice crystal. This process is
called riming. When an ice crystal grows to sizes larger than $2.5\ \unit{mm}$
through riming it is called a hailstone (Ch. 11, \textcite{lohmann16}). Snow flakes
consist of aggregates of ice crystals and range in sizes from millimeters to
centimeters.

The different types of hydrometeors and their respective sizes are summarized in
Fig.~\ref{fig:hydrometeors}. It should be noted that the size given in the
figure as well as the above paragraphs are only indications of size range
corresponding to each particle type. Their exact sizes can vary greatly from
particle to particle and from cloud to cloud.

\section{The physics of cloud formation}

Cloud hydrometeors form when water vapor contained in the air undergoes a phase
change from the gas phase to the liquid or frozen phase. This process is denoted
as condensation, for the change from the gas to the liquid phase, and
deposition, for the change from the gas to the ice phase.

A necessary condition for condensation to occur is that the air is
supersaturated with respect to liquid water. This means that the partial
pressure of water vapor exceeds the saturation vapor pressure with respect to
liquid water. Similarly, supersaturation with respect to the ice phase is
required for the formation of ice particles. The supersaturation required for
cloud formation  is reached when comparably warm and moist air is lifted
in the atmosphere. The resulting adiabatic cooling of the air leads to a
decrease in the saturation vapor pressures of water and ice and the air eventually
becomes supersaturated.

When water vapor reaches supersaturation with respect to either the liquid or
ice phase it enters a metastable state. This means that although the liquid
state is energetically favorable the transition is inhibited by an energy
barrier. Due to the random nature of the movements of water vapor molecules,
some of them eventually overcome the energy barrier by forming clusters of the
new stable phase inside the metastable gas phase. In the context of phase
transitions, this process of forming clusters of the stable state inside the
metastable parent state is referred to as nucleation. More specifically, two
types of nucleation are distinguished: Homogeneous nucleation refers to the
process of forming a new, pure cluster of molecules in the stable phase whereas
heterogeneous nucleation refers to the formation of a cluster of molecules in
the stable phase on or around a cluster of a different molecular species.

After a sufficiently large nucleus has formed, it will grow due to the
condensation or deposition of water molecules as long as its environment is
supersaturated with respect to its phase. Eventually, differences in fall
velocities between particles of different sizes will cause them to collide and
stick together.

Due to the different molecular properties of water and ice, slightly different
processes are involved in the formation of liquid cloud droplets and ice
particles. These together with the corresponding growth mechanism are explained
in more detail in the following two sections.

\section{Warm clouds}

As mentioned above, warm clouds are clouds that do not extend above
the $0\ \unit{^\circ C}$-isotherm and consist solely of liquid cloud droplets.

\subsection{Formation}

The formation of cloud droplets by homogeneous nucleation is highly unlikely due
to the height of the energy barrier separating the metastable gas phase from the
liquid phase. Instead, cloud droplets form through the activation of cloud
condensation nuclei (CCN). CCN are soluble aerosol particles, which take up
water molecules and grow even in environments that are not super-saturated. The
droplets which are formed by hygroscopic growth of aerosol particles are called
solution droplets (c.f. Fig.~\ref{fig:hydrometeors}). For sufficiently high
supersaturations, the energy barrier for the transition to larger cloud drops
vanishes leading to immediate condensation of all water molecules onto the
droplet which are available in its surroundings. The theory describing the
activation of CCN and their growth to cloud droplets is known as Köhler theory
\cite{kohler36}.

\subsection{Growth processes}

The condensation of water molecules onto the newly formed cloud droplet causes a
gradient in the concentration of water molecules initiating  a diffusive flow
of water vapor towards the droplet. The water vapor flowing towards the droplet
condenses onto it causing it to grow in size and mass. This process is called
growth by diffusion and condensation. The rate of diffusional growth decreases
with increasing droplet radius. For sizes larger than  $20\ \unit{\mu m}$ 
it has become so inefficient that another growth process takes over.

When the cloud droplets have grown sufficiently in mass, differences in fall
speed between droplets of different size or turbulence may cause droplets to
collide. If these droplets coalesce the resulting droplet will have grown
compared to the two colliding droplets. The newly-formed larger particle will
fall even faster through the cloud. Since heavier particles typically are
more efficient in collecting other cloud droplets, this illustrates why
collision-coalescence is a very efficient growth process. Only
collision-coalescence can explain the onset of rain only 20-30 minutes after the
formation of a cumulus cloud which can be observed in the atmosphere
(Ch. 7, \textcite{lohmann16}).

\section{Cold clouds}

Cold clouds extend  above $0\ \unit{^\circ C}$-isotherm and are characterized
by the presence of frozen hydrometeors.


\subsection{Formation}

In contrast to liquid droplets, both homogeneous and heterogeneous nucleation
are relevant for the formation of ice particles in the atmosphere. In agreement
with Oswald's rule of stages, homogeneous nucleation occurs only through the liquid
phase due to the prohibitively high energy barrier associated with the formation
of an ice nucleus directly from the vapor phase. This means that ice particles
are formed by homogeneous nucleation through the formation of an ice nucleus
inside an existing cloud or solution droplet and subsequent complete freezing of
the droplet. Nonetheless, the energy barrier for homogeneous nucleation of ice
particles remains so high that these processes occur only at
temperatures below $-36\ \unit{^\circ C}$ for solution droplets and
$-38\ \unit{^\circ C}$ for cloud droplets.

Heterogeneous nucleation of ice particles involves aerosol particles, so called
ice nucleating particles (INP), which provide a surface onto which the water
molecules can form aggregates with ice-like structure. Heterogeneous nucleation
is thought to occur both directly from the gas to the ice phase as well as
indirectly through an intermediate liquid phase particle. Heterogeneous
freezing, that is heterogeneous nucleation from the liquid phase, occurs when a
cloud droplet or a solution droplet comes in contact with an INP upon which the
droplet freezes. Heterogeneous freezing may also occur when water vapor
condenses directly onto the INP followed by freezing of the liquid nucleus
formed on the INP.

Alternatively, heterogeneous nucleation may occur directly from the vapor phase
by deposition of water molecules onto the INP. It is, however, still debated
whether this process really occurs directly from the vapor phase or whether an
intermediate liquid nucleus is formed on the INP.

\subsection{Growth processes}

In principle, the growth processes for ice particles are the same as for liquid
droplets. However, due the potential coexistence of particles in the liquid
phase these processes have slightly different characteristics as will be
explained below.

Due to the lower saturation vapor pressure of ice compared to that of water, a
newly-formed ice nucleus experiences a much higher ratio of supersaturation
after its formation than a cloud droplet would. This leads to a  diffusional
growth of ice crystals much faster rates than that of cloud droplets.
The rapid growth of the ice particles will deplete the surrounding air of water
vapor. This depletion may cause the environment to become sub-saturated with
respect to water while it remains saturated with respect to ice. If this is the
case, potentially present supercooled cloud droplets evaporate and their
molecules deposit onto the ice particles. This is known as the
Wegener-Bergeron-Findeisen process.

Similar as for cloud droplets, an ice crystal that grows sufficiently in mass
eventually starts to sediment. Size differences between different particles as
well as turbulence may cause particles to collide and potentially stick together
to form larger particles thus initiating growth by accretion. Accretion is the
general term for the growth of hydrometeors caused by the collision of two
particles resulting in a permanent union of the two particles. For ice particles,
growth by accretion can happen in two ways: The collision of two ice particles,
which is called aggregation, or the collision of an ice particle with a liquid
particle, which is called riming. Aggregation produces aggregates of snow
crystals that, if they don't melt, fall to the ground in the form of snow.
Particles produced by riming are known as graupel, when their diameter remains
below $2.5\ \unit{mm}$, and hail for sizes above that.

\subsection{Ice habits}

Ice crystals exhibit a fascinating range of different forms. Their crystal
structure depends on the thermodynamic conditions of their formation. This is
because the anisotropic surface tension of a newly formed ice nucleus depends
 on the temperature and supersaturation.

 It is common to distinguish three class of frozen hydrometeors: Pristine ice
 crystals, aggregates and rimed particles. Common shapes for ice crystals are
 plates and columns (c.f. Fig~\ref{fig:hydrometeors}). But also other shapes such as
 dendrites, stellar plates or needles can be observed. Snow aggregates are
 usually made up of 10 to 100 or more single crystals. They often consist of
 dendrites and thin plates. Finally, rimed particles are typically spherical
 with densities slightly lower than that of solid ice due air inclusions.

\section{Precipitation}

Precipitation occurs when the hydrometeors inside a cloud have grown
sufficiently for their fall speeds to exceed the updraft velocity. Due to the
ineffectiveness of diffusional growth for cloud droplets and ice crystals at
larger particle sizes, growth by collision-coalescence and growth by accretion
are required to form precipitation at the rates observed in the atmosphere.

The exception to this rule is a phenomenon called cloud-less ice precipitation
which refers to ice particles that grow large enough to sediment and reach
the ground. This, however, occurs only in very cold climates with clean air such
as the Arctic and is therefore not relevant at global scales.

Since the transition between suspended cloud hydrometeors and precipitating
hydrometeors is continuous no explicit distinction will be made in the following
between the two. Instead ice hydrometeors will be considered as a single species
of particles include ice crystals, snowflakes, graupel and hail.

\chapter{Microwave radiative transfer in the atmosphere}
\label{ch:radiative_transfer}

The underlying physical mechanism that allows cloud and precipitation to be
remotely sensed is their interaction with electromagnetic radiation that can be
measured from afar using suitable detectors. These interactions are described by
the theory of radiative transfer. Since this theory is essential for the
understanding and development of retrieval methods, this section provides an
introduction to radiative transfer of microwaves in the atmosphere.
The focus is put on the interaction of radiation with clouds. This
presentation is mostly based on the more comprehensive texts by
\textcite{thomas02, mischenko02, wallace06}.

\section{The theory of radiative transfer}

Radiative transfer theory describes radiation as monochromatic beams that
transport radiant energy through the atmosphere. One of the fundamental
quantities of the theory is the spectral intensity $I_\nu$ defined as the rate
at which a beam consisting of radiation from an infinitesimal frequency interval
$d\nu$ centered at $\nu$, with angular extent $d\omega$ and propagating into
direction $\hat{\vec{n}}$, transports energy through an infinitesimal area $dA$:
\begin{align}\label{eq:spectral_intensity}
  I_\nu = \frac{d^5E}{\cos(\theta) dA\ dt\ d\omega\ d\nu}
\end{align}
where $\cos(\theta)$ is the incidence angle between the surface normal of $A$
and the direction of propagation of the beam. In addition to that, a
monochromatic beam of radiation has a polarization state, which describes how
the energy flux is split up between the two components of the electric field
perpendicular to the propagation direction as well as their respective phase.
The intensity of a beam and its polarization state are described by the Stokes
vector
\begin{align}
  \vec{I} &= \left [ \begin{array}{c}I_\nu \\ Q \\ U \\ V\end{array} \right ]
\end{align}
The four components $I_\nu, Q, U$ and $V$ of the Stokes vector fully characterize an
electromagnetic plane wave to the extent that it can be measured using
traditional detectors. This means that all  measurable quantities of the
radiation can be derived from the corresponding Stokes vector.

The Stokes vector can be directly related to the
electromagnetic field strength of an electromagnetic plane wave allowing it to
be derived from the more fundamental theory of electromagnetism. The key
advantage of radiative transfer theory, however, is that it allows a simplified
treatment of the problems relevant to atmospheric remote sensing which are too
complex to be solved by direct application of the laws of electrodynamics.

\subsection{Interactions with matter}

The Stokes vector provides a full description of the radiation measured by any
passive remote sensing instrument. To model the radiation reaching the detector,
a suitable description how this radiation is created as well as how it changes
as it propagates through the atmosphere is required. A common approach in
radiative transfer theory is to distinguish three fundamental types of such
interactions of radiation with matter: The emission of radiation, its
absorption, and the scattering of radiation away from and into its propagation
path.

\subsubsection{Emission}

At temperatures above absolute zero, all matter emits radiation through the
process of thermal emission. Thermal emission occurs when matter transitions
from a quantum mechanical state of higher energy to one of lower energy which
causes the surplus of energy to be emitted in the form of radiation. When
considering radiation in the lower atmosphere, the relevant emitters of
radiation are the ocean or land surface as well as gas molecules or suspended
particles.

A fundamental concept for the description of emission is that of a black body.
A black body is a piece of matter that absorbs all incoming radiation. At a
given temperature $T$, the emission of a black body is isotropic and un-polarized.
Its spectral intensity is given by Planck's law:
\begin{align}
  B_\nu(\nu, T) &= \frac{2h\nu^3}{c^2}\frac{1}{\exp ( \frac{h\nu}{k_B T} )- 1}
\end{align}
where $c$ is the speed of light in the medium, $h$ is the Planck constant and
$k_B$ is the Boltzmann constant. The Stokes vector describing the emission from
a black body is given by
\begin{align}
  \vec{I} &= \left [ \begin{array}{c}B_\nu \\ 0 \\ 0 \\ 0\end{array} \right ].
\end{align}

The concept of the black body is used to define the emission from other forms
of matter using the emissivity vector $\vec{\epsilon}$:
\begin{align}\label{eq:emissivity}
 \vec{I} &= \vec{\epsilon} \cdot B_\nu
\end{align}
The main difference between the treatment of emission from a volume compared
to that of a surface is the unit of the emission vector $\vec{\epsilon}$. For
a volume, it is defined per unit length of the path through the volume, while
for a surface this is not necessary.

Due to the distinct orientation that surfaces have with respect to the viewing
geometry, the emissivity vector generally depends on the emission angle. For
particles this is generally also the case, but since most particles in the
atmosphere are randomly oriented it is often neglected. Although black-body
radiation is unpolarized, emission from general emitters can be polarized. An
important example is the ocean surface, which is highly polarized around the
Brewster angle at $53^\circ$.

\subsubsection{Absorption}

Absorption refers to the process of radiation being converted into internal
energy of the matter it interacts with. Mathematically, this process is described
by the absorption vector $\vec{\alpha}$, defined as the fraction of the incoming
radiation that is absorbed along an infinitesimal distance $ds$ along the
propagation path:
\begin{align}
\vec{I}_\text{absorbed} &= (\vec{\alpha} \cdot\ ds) \odot \vec{I}
\end{align}
Here $\odot$ denotes the element-wise product of the absorption vector and
the Stokes vector $\vec{I}$ of the incoming radiation. Absorption may be
understood as the inverse process of thermal emission. Formally, this is
expressed by Kirhoff's  law of radiation
\begin{align}
  \vec{\alpha} &= \vec{\epsilon},
\end{align}
which states that the absorption vector is identical to the emissivity
vector defined in Eq.~\ref{eq:emissivity}.
This law is applicable to all matter in the atmosphere given that it is in a
state of local thermal equilibrium (LTE). LTE occurs when the density of matter
is sufficiently high so that the population rates of energy states above the
ground state are determined by thermal collisions rather than the absorption of
radiation. This decouples the emission of radiation from the radiation field,
allowing the simplified treatment of matter as thermal emitters with the
emission rates independent of the radiation field. LTE is a valid assumption for
radiative transfer in the lower atmosphere.

\subsubsection{Scattering}

When a beam of radiation impinges upon a particle, their interaction may cause a
deviation of parts of the beam from the original propagation path. To first
order, scattering decreases the intensity of the beam. This particular process
is referred to as single scattering. As it propagates through the atmosphere,
the intensity of a beam is decreased by the effects of absorption and single
scattering. The combination of these two processes is referred to as attenuation
or extinction. As the rate of scattering increases, also the effect radiation
that is being scattered into the beam has to be taken into account.

Mathematically, the scattering of a beam of light propagating in direction
$\vec{n}$ into the direction $\vec{\hat{n}}$ is described by the phase
matrix $\mat{Z}(\vec{\hat{n}}, \vec{n})$:
\begin{align}
  \vec{I}_\text{scattered}(\vec{\hat{n}}) &= \mat{Z}(\vec{\hat{n}}, \vec{n}) \vec{I}(\vec{n})
\end{align}
The combined, attenuating effects of scattering and absorption are given by
the attenuation matrix $\mat{K}$, which is the sum of the absorption vector
$\vec{\alpha}$ and the fraction of radiation scattered away from the propagation
path:
\newcommand*{\vertbar}{\rule[-1ex]{0.5pt}{2.5ex}}
\newcommand*{\horzbar}{\rule[.5ex]{2.5ex}{0.5pt}}
\begin{align}
  \vec{K} &=
  \left [ \begin{array}{cccc}
      \vertbar & \vertbar & \vertbar & \vertbar \\
      \vec{\alpha} & \vec{0} & \vec{0} & \vec{0} \\
      \vertbar & \vertbar & \vertbar & \vertbar
    \end{array} \right ]
       + \int_{\vec{\hat{n}}} d\vec{\hat{n}}\ \mat{Z}(\vec{\hat{n}}, \vec{n})
\end{align}

\subsection{The radiative transfer equation}

The previous section introduced the fundamental interactions of radiation
with matter and how they are described mathematically in radiative transfer
theory. Combining the three processes of emission, absorption and scattering,
the change that a beam undergoes as it travels a distance $ds$ along its
propagation path through the atmosphere is described the vector radiative
transfer equation (VRTE):
\begin{align}\label{eq:vrte}
  \frac{d\vec{I}(\vec{n})}{ds} &=
  -\mat{K}\vec{I}(\vec{n}) + \vec{\alpha} \cdot B_\nu(T) + \int_{\hat{\vec{n}}} d\hat{\vec{n}} \ \mathbf{Z}(\vec{n}, \vec{\hat{n}}) \vec{I}(\vec{\hat{n}}).
  \end{align}
The radiation field for an arbitrary atmosphere can be computed by solving
Equation~(\ref{eq:vrte}). What is required for this are the values of
temperature, absorption vector $\vec{\alpha}$ and phase matrices $\mat{Z}$
throughout the atmosphere as well as a suitable method for solving the radiative
transfer equation. The values of $\vec{\alpha}$ and $\mat{Z}$ describe how a
specific volume element of the atmosphere absorbs and scatters radiation. Their
values therefore depend on the concentrations of gases and particulate matter in
the atmosphere.

Approximate values of $\vec{\alpha}$ and $\mat{Z}$ for different materials in
the atmosphere can be measured experimentally or in special cases even derived
from first principles. Typically they depend on local properties of the
atmosphere such as temperature, pressure or concentration of gases or particles.
Numerical models for $\vec{\alpha}$ and $\mat{Z}$ together with the VRTE thus
allow the radiation field observed by remote sensing intruments to be related to
the state of the atmosphere. Together with the methods described in
Section~\ref{ch:inverse_problems}, this forms the basis of atmospheric
remote sensing.


\section{Microwave observations of clouds and precipitation} 

To illustrate the application of radiative transfer theory to observations of
clouds and precipitation, we now turn back to the example cloud scene considered
in Sec.~\ref{sec:observations}. A simple hydrometeor retrieval has been
performed to estimate the mass concentrations of frozen and liquid hydrometeors
in the observed cloud using the radar observations displayed in
Fig.~\ref{fig:modis_cloudsat}. For simplicity, it was assumed that all
hydrometeors below (in altitude) the $0\ \unit{^\circ C}$-isotherm are in the
liquid phase and the ones above in the ice phase. The retrieved mass
concentrations are displayed in Fig.~\ref{fig:retrieval}. Since a standard
tropical atmosphere has been assumed to retrieve the cloud properties, the
melting layer and hence the boundary between liquid and ice hydrometeors is at a
constant altitude across the whole scene. Although this is certainly not a very
accurate assumption especially since the radar reflectivities show signs of
convective activity, it is sufficient to illustrate the basic interaction of
microwave radiation with clouds and precipitation. The simplified cloud model
obtained in this way is can now by used to simulate the signatures of liquid and
frozen hydrometeors in the passive microwave observations.

\begin{figure}
\includegraphics[width = \textwidth]{notebooks/retrieval.png}
\caption{Mass content of frozen (blue) and liquid hydrometeors retrieved
from the radar reflectivities show in Fig.~\ref{fig:modis_cloudsat}}
\label{fig:retrieval}
\end{figure}

\subsection{Liquid hydrometeors}

We start by investigating the effect of liquid hydrometeors on passive microwave
observations. For this, all frozen hydrometeors in the scene are ignored and
observations are simulated of the four passive microwave frequencies considered
in Fig.~\ref{fig:modis_cloudsat}. The results of these simulations are displayed
in Fig.~\ref{fig:signals_rain}. Shown is the signal from clouds and precipitation
defined as the difference in the simulated microwave signal with respect to a cloud-free
reference observation:
\begin{align}
  \Delta y &= y_\text{cloudy} - y_\text{clear}
\end{align}
At frequencies as low as $10\ \unit{GHz}$, the signal observed from the rain is positive.
At these frequencies, rain drops interact with radiation mostly through absorption. Above
the sea surface, which acts as a cold background, the rain is therefore observed as a
warm signal. At $37\ \unit{GHz}$, an even stronger positive signal is observed from the
precipitation. The increase in the strength of the signal is due to the increased absorption
at shorter wavelengths.
As the frequency is increased to $89\ \unit{GHz}$, the precipitation signal switches sign
from  positive to negative. This is because  the sensitivity to water vapor
causes the background to become warmer, while at the same time the rain drops become
more effective scatters. The observed rain signal is therefore due to radiation that is being
scattered away from the line of sight by the rain drops, leading to a negative signal.

At $183\ \unit{GHz}$, finally, the precipitation signal has fully disappeared.
This is because these frequencies the sensitivity to water vapor is so high that
the sensor is essentially blind to the lower parts of the troposphere in which
the rain is located.

\begin{figure}
\includegraphics[width = \textwidth]{notebooks/signals_rain.pdf}
\caption{Simulated differences in observed passive radiances with respect to a
  cloud-free reference observation $\Delta y$ for different microwave
  frequencies when only liquid hydrometeors are considered.}
\label{fig:signals_rain}
\end{figure}

\subsection{Frozen hydrometeors}

To now assess the additional effects of frozen hydrometeors, the simulations are run once
again but this time including also the frozen hydrometeors in the simulations. The resulting
cloud signals are displayed in Fig.~\ref{fig:signals_ice}.

First of all, it can be noticed that the precipitations signals observed at
$10\ \unit{GHz}$ and $37\ \unit{GHz}$ are not affected by the presence of frozen
hydrometeors. Although frozen precipitating particles are of similar sizes as
rain drops, differences in their dielectric constants cause the ice to interact
less strongly with the radiation than water. At $89$ and $183\ \unit{GHz}$,
however, a clear negative signal from the ice particles is observed. At
$183\ \unit{GHz}$, all of the observed cloud signal is due to the scattering
from frozen hydrometeors.

\begin{figure}
\includegraphics[width = \textwidth]{notebooks/signals_ice.pdf}
\caption{Simulated differences in observed passive radiances with respect
  to a cloud-free reference observation $\Delta y$ when both liquid and frozen
  hydrometeors are considered. Dashed lines show the corresponding signal due to
  rain only (c.f. Fig.~\ref{fig:signals_rain}).}
\label{fig:signals_ice}
\end{figure}

\subsection{Sub-millimeter wavelengths}

Finally, the simple cloud model can be used to demonstrate the benefits of
sub-millimeter wavelengths for cloud observations. For this, we consider the
cloud signal observed by four of the channels of the upcoming ICI instrument
located at $248, 325, 448$ and $664\ \unit{GHz}$. The simulated cloud signals
are displayed in Fig.~\ref{fig:signals_ici}. The cloud signal in all of the
considered channels is considerably stronger than that at $183\ \unit{GHz}$. In
particular in the left part of the scene, the cloud signals at sub-millimeter
wavelength are significantly stronger than that observed at $183\ \unit{GHz}$.
Since observations at $183\ \unit{GHz}$ are sensitive only to particles large
enough to be considered snow, the signal observed here indicates the presence
ice clouds. This shows that ICI will bring immense benefits for observations
of clouds using microwave radiation.
\begin{figure}
\includegraphics[width = \textwidth]{notebooks/signals_ici.pdf}
\caption{Signature of clouds in microwave observations. The plot shows the
  simulated cloud signal as it would be observed by the
  reference observation when both liquid and frozen hydrometeors are
  considered.}
\label{fig:signals_ici}
\end{figure}

%  d8Nu.  9888c
%  88888  98888
%  "***"  9888%
%       ..@8*"
%    ````"8Weu
%   ..    ?8888L
% :@88N   '8888N
% *8888~  '8888F
% '*8"`   9888%
%   `~===*%"`
%
%%%%%%%%%%%%%%%%%%%%%%%%%%%%%%%
%%%%%%%%%%%%%%%%%%%%%%%%%%%%%%%
\chapter{Inverse problems}
\label{ch:inverse_problems}

The previous chapters provided an overview over the general properties of clouds
and their interaction with electromagnetic radiation which is used to observe
them. This chapter describes the mathematical methods that are used to infer
properties of clouds from their signatures in the observed electromagnetic
radiation. Mathematically, this task is formulated as an inverse problem. This
is because inferring the cloud properties from observations can be viewed as the
inverse of the problem of predicting the observations given the cloud
properties, which is referred to as the forward problem. How the forward problem
can be solved using radiative transfer theory has been described in the previous
chapter. In this chapter we now turn to the task of solving the inverse problem.

\section{Formulation}

Mathematically, the general inverse problem of remote sensing is formulated as
follows: Let $\vec{x} \in \mathbb{R}^n$ be an arbitrary vector that describes
the state of the atmosphere. The vector space of all possible states is referred
to as the state space and for simplicity assumed to be given by $\mathbb{R}^n$.
The state atmospheric state $\vec{x}$ is observed through an observations
system, which produces the observation vector $\vec{y} \in \mathbb{R}^m$.
Furthermore, it is assumed here that a forward model $\mathbf{F}: \mathbb{R}^n
\rightarrow \mathbb{R}^m$ exists that allows computing the observation $\vec{y}
= \mathbf{F}(\vec{x})$ corresponding to any given state vector $\vec{x}$. The
inverse problem consists of determining the state vector $\vec{x}$ corresponding
to a given the observation vector $\vec{y}$, hence to invert the forward model
$\mathbf{F}$.

The general difficulty with inverse problems is that they do not admit
a unique solution. This is because, at least in atmospheric remote sensing, the
problem is generally underconstrained. This means that the amount of information
in the observations $\mathbf{y}$ is not sufficient to uniquely determine a state
$\mathbf{x}$. Examples are different cloud configurations that result in the
same measurement vector such as for example a low-level cloud covered by an
opaque high-level cloud. From the measurement vector $\vec{y}$ alone it is
impossible to determine a unique state $\vec{x}$ as it will be the same
independent of the presence or properties of the low-level cloud.

Simultaneously to being underconstrained, the problem may be overconstrained.
This happens when different components of the measurement vector provide seemingly
contradictory information on the measurement state $\vec{x}$ due errors random
errors in $\vec{y}$.

\section{Solution}

A common approach in atmospheric remote sensing to solve inverse problems is
the application of Bayesian statistics. This means that instead of searching a
unique solution to the inverse problem, the solution is found in the form of a
probability distribution that describes how likely it is that any of the
elements of the state space has produced a given observation $\vec{y}$.

The approach that will be presented in the following is known as the optimal estimation
method (OEM, \textcite{rodgers00}). The method makes three basic assumptions in order
to solve the inverse problem:
\begin{enumerate}
\item That the forward model $\mathbf{F}$ is linear or at most weakly non-linear,
\item that the knowledge available about $\vec{x}$ can be described by a Gaussian distribution,
  with mean $\vec{x}_a$ and covariance matrix $\mat{S}_a^{-1}$
\item that the errors affecting $\mathbf{y}$ are Gaussian with covariance matrix $\mat{S}_\epsilon$.
\end{enumerate}

Under these assumptions, both the a priori distribution for  $\vec{x}$ as well as the conditional
probability of observing the measurement $\vec{y}$ given the state $\vec{x}$ are Gaussian:
\begin{align}
p(\vec{x}) &= \frac{1}{(2 \pi)^{\frac{-n}{2}} \text{det}(\mat{S}_a)^{-\frac{1}{2}}} \exp \left \{ -\frac{1}{2}(\vec{x} - \vec{x}_a)^T\mat{S}_a^{-1}(\vec{x} - \vec{x}_a ) \right \}\label{eq:px}\\
p(\vec{y} | \vec{x}) &= \frac{1}{(2 \pi)^{\frac{-m}{2}} \text{det}(\mat{S}_e)^{-\frac{1}{2}}} \exp \left \{-\frac{1}{2}(\vec{x} - \mathbf{F}(\vec{x}))^T\mat{S}_e^{-1}(\vec{x} - \mathbf{F}(x)) \right \}\label{eq:pyx}
\end{align}

In the Bayesian framework the solution of the inverse problem is simply the a posteriori distribution
$p(\vec{x} | \vec{y})$ of $\vec{x}$ given the observation vector $\vec{y}$. It is found by applying
Bayes theorem
\begin{align}
  p(\vec{y} | \vec{x}) &= \frac{p(\vec{y} | \vec{x})p(\vec{x})}{p(\mathbf{y})} \\
  &\propto  p(\vec{y} | \vec{x})p(\vec{x})
\end{align}
to the probabilities  (\ref{eq:px}) and (\ref{eq:pyx}).

As a specific solution of the retrieval problem, generally the most likely state is chosen,
denoted as the maximum a posteriori (MAP) estimator for $\vec{x}$. It can be found by minimizing
the log-likelihood of the posterior distribution, which has the form:
\begin{align}\label{eq:oem_cost}
  -\mathcal{L} \propto (\vec{F}(\vec{x}) - \vec{y})^T\mat{S}_\epsilon^{-1}(\vec{F}(\vec{x}) - \vec{y})
  + (\vec{x} - \vec{x}_a)^T \mat{S}_a^{-1} (\vec{x} - \vec{x}_a)
\end{align}

Solving the retrieval problem has thus been reduced to minimizing the negative log-likelihood of
the posterior distribution. When the forward model is non-linear, minimizing
Eq. (\ref{eq:oem_cost}) must be performed iteratively using suitable optimization methods such
as the Gauss-Newton or Levenberg-Marquardt methods \cite{boyd04}.

\section{Error estimation}

A major advantage of the OEM formalism is that it allows precise characterization of
the errors that affect the retrieved state. Given below is a derivation of the retrieval
error from the formulation of the OEM for a linear forward model and in the absence of
forward model errors. How these results can be generalized to a non-linear forward model
and the presence of forward modeling errors are outlined afterwards.

\subsection{The idealized case}

We start by defining a retrieval operator $\hat{\vec{x}} = \mathbf{R}(\vec{y})$
which represents the application of the retrieval to an observation vector
$\vec{y}$ yielding the retrieved state $\hat{\vec{x}}$. If the forward model is
assumed to be exact, then $\vec{y}$ can be written as the sum of the forward
model evaluated at the true state $\vec{x}$ and a random vector of measurement
noise:
\begin{align}
  \mathbf{y} = \mathbf{F}(\vec{x}) + \vec{\epsilon}.
\end{align}
Inserting this into $\mathbf{R}$ and linearizing the forward model $\mathbf{F}$ about the a
priori state $\vec{x}_a$ yields
\begin{align}
  \mathbf{R}(\vec{y}) &= \mathbf{R} \left (\mathbf{F}(\vec{x}_a) + \mathbf{K}(\vec{x} - \vec{x}_a) + \vec{\epsilon} \right),
\end{align}
where $\mat{K} = \frac{d\mat{F}}{d\vec{x}}$ is the Jacobian of the forward model.
Now, linearizing the retrieval operator  about $\mathbf{y}$ allows us to write
\begin{align}
 \vec{\hat{x}}  &= \mathbf{R}(\mathbf{F}(\vec{x}_a)) +  \frac{d\mathbf{R}}{d\vec{x}}\mathbf{K}(\vec{x} - \vec{x}_a) + \frac{d\mathbf{R}}{d\vec{x}} \epsilon.
\end{align}
Assuming an unbiased retrieval operator, i.e. $\mathbf{R}(\mat{F}(\vec{x})) - \vec{x}_a = 0$, the retrieved
information on the state $\vec{x}$ may be written as
\begin{align}
\hat{\vec{x}} - \vec{x}_a &=  \frac{d\mathbf{R}}{d\vec{x}}\mathbf{K}(\vec{x} - \vec{x}_a) + \frac{d\mathbf{R}}{d\vec{x}} \epsilon.
\end{align}
The derivative of the retrieval operator $\mathbf{R}$ with respect to the observation vector $\mathbf{y}$
is referred to as the gain matrix $\mathbf{G}$ and is found to be given by
\begin{align}
  \mathbf{G} &= (\mat{K}^T\mat{S}_\epsilon^{-1}\mat{K} + \mat{S}_a^{-1})^{-1} \mat{K}^T\mat{S}^{-1}_\epsilon.
\end{align}
We further define the so-called averaging kernel matrix $\mathbf{A} = \mathbf{G}\mathbf{K}$. With
this, the retrieval error can be written as
\begin{align}\label{eq:retrieval_error}
  \hat{\vec{x}} - \vec{x} &=  \underbrace{(\mathbf{A} - \mat{I})(\vec{x} - \vec{x}_a)}_{\text{Smoothing error}}
  + \underbrace{\frac{d\mathbf{R}}{d\vec{x}} \epsilon}_{\text{Error due to noise}}.
\end{align}
This shows that the retrieval error can be written as the sum of two
contributions. The first term on the right hand side is the so called smoothing
error. It can be interpreted as the retrieval error that occurs because of the
limited resolution of the observation system. Its covariance matrix is given by
\begin{align}
  \mat{S}_s = (\mat{A} - \mat{I})\mat{S}_\epsilon(\mat{A} - \mat{I}).
\end{align}
The second term on the right-hand side of Eq.~(\ref{eq:retrieval_error}) is the error caused by the noise in the observations. Its covariance matrix is given by
\begin{align}\label{eq:retrieval_noise}
\mat{S}_m &= \mat{G}_y\mat{S}_\epsilon\mat{G}_y^T.
\end{align}

\subsection{Handling forward model error and non-linearity}

The derivation presented above assumed an ideal forward model, which is of
course rarely the case in reality. It is possible to generalize the formulation
of the retrieval error, if the forward model error can be assumed to be bias
free and described using a Gaussian distribution with covariance matrix
$\mat{S}_e$. In this case the general retrieval error can be obtained by simply
replacing $\mat{S}_\epsilon$ in Eq. (\ref{eq:retrieval_noise}) with
$\mat{S}_\epsilon + \mat{S}_e$.

If the forward model in non-linear, the forward model in the above derivation has to be
linearized about the most recent state in the retrieval iteration $\vec{x}_i$. In this
case the bias term $\mathbf{R}(\mat{F}(\vec{x}_i)) - \vec{x}_a$ can no longer be assumed
to be zero, and the retrieval results will become biased.

\chapter{Summary of appended papers and outlook}
\label{ch:appended_papers}

After introducing the physical and mathematical principles of the remote sensing
of clouds, this chapter now turns towards the research that has been carried out
under the general aim of preparations for the upcoming ICI sensor. The two
scientific articles appended to this thesis investigate the concept of combining
radar and radiometer observations for retrieving ice hydrometeors. The studies
were carried out within the context of the study ``Scientific Concept Study for
Wide-Swath High-Resolution Cloud Profiling'' funded by the European Space Agency
which investigated the potential benefits of a hypothetical cloud radar mission
flying in constellation with ICI to provide co-located radar and sub-millimeter
radiometer observations.

\section{Paper A: Synergistic radar and radiometer retrievals of ice hydrometeors}

A synergistic retrieval algorithm is proposed which uses the OEM to retrieve ice
hydrometeors from combined radar and sub-millimeter radiometer observations. The
study aims to establish the fundamental synergies of the combined observations,
i.e. the additional information that can be gained from the observations when
they are used simultaneously in the retrieval instead of considering them
separately.

\subsection{Data and methods}

The study uses simulated observations from a high-resolution atmospheric model
to produce synthetic, co-located observations from a cloud radar and a
sub-millimeter radiometer. By applying the retrieval to these synthetic
observations its performance is  assessed. The combined retrieval is compared
to radar-only and passive-only versions of the retrieval algorithm to establish
the advantages of the synergistic retrieval approach.

\subsection{Results}

The results obtained in this study show the combination of radar and radiometer
observations helps to better constrain the microphysical properties of clouds.
The increased sensitivity to microphysical properties of the cloud reduces
uncertainties in retrieved ice mass concentrations. In addition to that, the
combined retrieval showed improved skill in the detection and retrieval of
liquid clouds.

\subsection{Conclusions}

The increased information content of the combined observations allows an
additional degree of freedom of the distribution of ice hydrometeors to be
retrieved, which helps to reduce uncertainties in the retrieved ice mass
concentrations. This information gain could be attributed to the sub-millimeter
channels of the ICI sensor. It is therefore concluded that combined retrievals
involving radar and sub-millimeter radiometer observations are a promising
approach that may help to reduce uncertainties in retrievals of frozen
hydrometeors.

\section{Paper B: Relating microphysical and radiometric properties of cloud
hydrometeors at millimeter and sub-millimeter wavelengths}

The second paper builds upon the first one by applying the synergistic retrieval
algorithm to observations from a recent flight campaign. One of the few
currently available sensors that can produce ICI-type sub-millimeter
observations of clouds is the International Submillimetre Airborne Radiometer
(ISMAR, \textcite{fox17}), which serves as the airborne demonstrator for the ICI
sensor. In 2016, ISMAR took part in a joint flight campaign in which three
research aircraft performed a simultaneous overpass of a mid-latitude cloud
system and observed it using a wide range of remote sensing sensors. Since cloud
radars were present on the other two aircraft, the observations made during the
campaign provide a unique opportunity to apply and test the synergistic
retrieval algorithm.

\subsection{Data and method}

The study uses observations acquired by the joint flight of the High Altitude
and Long Range Research Aircraft (HALO), the Facility for Airborne Atmospheric
Measurements (FAAM) and the Service des Avions Francais Instrumentions pour la
Recherche en Environnement (SAFIRE) research aircraft. The data consists of
remote sensing observations from the overpass and in-situ data collected by the
FAAM aircraft.

The in-situ data available from the flight are used to characterize the
microphysical properties of the hydrometeors in the observed cloud. Since
assumption on these properties affect the results of the retrieval, this
characterization is necessary for their analysis.

Following this, the synergistic retrieval algorithm is applied to the
observations from the flight. The fits of the forward model to the observations
are evaluated to check the consistency of the radiative transfer modeling upon
which the retrieval is based. The retrieved hydrometeor profiles are compared to
those derived from the in-situ data. Finally, the consistency of the retrieved
hydrometeor distributions with the other observations from the flight is assessed
by comparing them to simulations obtained from the retrieval results.

\subsection{Results}

An important result of this study is that the retrieval does achieve a good fit
to the observations over large parts of the scene, which shows that the applied
radiative transfer scheme is able to consistently model the observations over
the wide range of microwave frequencies that was considered here. Two regions in
which the retrieval does not fit the observations show signs of convective
activity indicating the presence of a signal from the specific microphysical
properties of the hydrometeors in the updraft.

Although dependent on the employed particle model, the retrieval results show
generally good agreement with the in-situ measurements. This dependency,
however, is found to be consistent with the microphysical characterization
derived from the in-situ measurements, which speaks in favor of the validity of
the retrieval method. Moreover, the simulated observations of an additional
radar operating at a different frequency show good agreement with the real
observations.

\subsection{Conclusions}

The results presented in this study validate the implementation of the developed
synergistic retrieval algorithm and thus confirm the results from the first
study. Moreover, the good fit to the observations obtained in the retrieval
demonstrates the consistency of the radiative transfer modeling through clouds
at millimeter and sub-millimeter wavelengths. The results therefore provide an
important validation case for the modeling of radiative transfer at
sub-millimeter wavelengths.

\section{Outlook}

\subsection{Relevance of the results}

The focus of the presented research were combined retrievals using radar and
sub-millimeter radiometer observations. Since such observation are currently not
available from any ongoing satellite missions, this research is relevant to
prospective satellite missions involving radar and sub-millimeter observations.
In addition to that, the second study demonstrated the usefulness of the
approach also for the application in field campaigns where it may be used to
validate the radiative transfer modeling of clouds and possibly even to study
their microphysical properties.

The presented results also have a more general relevance with respect to the
upcoming ICI mission. Since currently available sub-millimeter observations of
clouds are limited, the radiative transfer modeling at these frequencies is
still afflicted with considerable uncertainties. These uncertainties will need
to be addressed in order to put the observations from ICI to proper use. The
application of the combined retrieval described in Paper B validated the
radiative transfer modeling at sub-millimeter wavelengths but also highlighted
remaining modeling issues regarding the properties of hydrometeors in regions of
strong convective activity.

\subsection{Future work}

Although the presented work established the basic consistency of radiative
transfer modeling of clouds at sub-millimeter wavelengths, there remain a number
of challenges that should be addressed. One of them is certainly the
representation of the ice particles in radiative transfer simulations at
sub-millimeter wavelengths. This could be further investigated by making use of
other field campaigns involving the ISMAR radiometer. Alternatively,
co-locations of currently available satellite observations could be used as has
been attempted in \citet{ekelund19}. This work could be further extended by
including infrared observations and by using directly co-located observations
such as the ones shown in Sec.~\ref{sec:observations}.

A more specific line of future work that arises from the results of the second
study is to investigate the radiometric properties of hydrometeors in convective
updrafts. The observations from the flight campaign could serve as a case study
to establish the basic radiometric properties of the hydrometeors present in
these updrafts. These could then be used to investigate the effect of the
derived properties on already available microwave observations of clouds.

Finally, since the ability to model the radiative transfer at sub-millimeter
wavelengths is the foundation for understanding and making use of the observations
that will be provided by ICI, the general consolidation of radiative transfer models at
sub-millimeter wavelengths remains an essential issue. An important resource for this
are the air-borne observations from the ISMAR sensor, which should be used for continued
validation studies for the radiative transfer modeling at these wavelengths.
